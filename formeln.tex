\documentclass[landscape,8pt]{scrartcl}

\usepackage[utf8]{inputenc}
\usepackage[T1]{fontenc}
\usepackage{fourier}
\usepackage[ngerman]{babel}
\usepackage{amsmath}
\usepackage{xcolor}
\usepackage[margin=0.3in]{geometry}
\usepackage{multicol}
\setlength{\columnseprule}{0.4pt}

\newcommand\hr{\par\vspace{-.5\ht\strutbox}\noindent\hrulefill\par}
\newcommand{\mysection}[1]{\hr\section{\fcolorbox{white}{green}{#1}}}
\newcommand{\mysubeqsec}[1]{\subsection{\fcolorbox{white}{cyan}{\ensuremath{#1}}}}
\newcommand{\mysubsubeqsec}[1]{\subsubsection{\fcolorbox{white}{yellow}{\ensuremath{#1}}}}
\newcommand{\mysubsection}[1]{\subsection{\fcolorbox{white}{cyan}{#1}}}
\newcommand{\myeq}[1]{\begin{equation*}#1\end{equation*}}
\newcommand{\bmyeq}[1]{\begin{equation*}\boxed{#1}\end{equation*}}
\newcommand{\mydoubleeq}[2]{\begin{align*}\boxed{#1} & \hfill & \boxed{#2}\end{align*}}

\newcommand{\dx}{\ensuremath{\,\mathrm{d}x}}
\newcommand{\du}{\ensuremath{\,\mathrm{d}u}}
\newcommand{\dv}{\ensuremath{\,\mathrm{d}v}}
\newcommand{\myto}{\colorbox{blue!20}{\ensuremath{\longmapsto}}}

\hyphenation{
	Sch-ran-ke
	Kri-ter-i-um
	Lei-b-ni-tz
}

\begin{document}

\begin{multicols}{4}

%\twocolumn


\mysection{Funktionen}

\mysubsection{Surjektivität}

Definiere $y = f(x)$. Löse nach $x$ auf. Ist jedes $y$ in der
Gleichung in Definitionsmenge vertreten? Wenn ja, ist die Funktion surjektiv.

\mysubsection{Injektivität}

Nehme an, dass $f(x_1) = f(x_2)$. Versuche zu zeigen, dass daraus folgt, dass
$x_1 = x_2$. Dann ist die Funktion injektiv.

\mysubsection{Bijektivität}

Eine Funktion ist bijektiv, wenn sie injektiv und surjektiv ist.

\mysubsection{Funktion Invertieren}

$y = f(x)$ nach $x$ auflösen, dann $y$ und $x$ vertauschen.

\mysection{Folgenmonotonie}

Sei $a_n$ eine Folge, dann ist $a_n$ streng monoton fallend, wenn
$a_{n + 1} > a_n$ bzw. streng monoton steigend, wenn 
$a_{n + 1} < a_n$ (bzw. \textit{nur} monoton steigend/fallend,
wenn $\geq$ oder $\leq$).

\mysubsection{Monotoniekriterium}

Jede monotone und beschränkte Folge konvergiert.

\mysubsection{Nullfolgenkriterium}

Eine Folge $a_k$ kann nur konvergieren, wenn $\lim_{n\to\infty} a_n = 0$.

\mysection{Konvergenz}

\mysubsection{Konvergenzradius}

\myeq{r = \lim \left|\frac{a_n}{a_{n+1}}\right|}

\mysubsection{Absolute Konvergenz}

$\displaystyle \sum_{k = 0}^\infty a_k$ heißt \textit{absolut konvergent},
wenn $\displaystyle \sum_{k = 0}^\infty |a_k|$ konvergiert. Ist eine Folge
absolut konvergent, dann ist sie auch konvergent.

Die Reihe $\sum_{n=1}^\infty \frac{(-1)^{n+1}}{n}$ ist konvergent, aber nicht
absolut konvergent.

\mysubsection{Majorantenkriterium}

Sei $a_k$ eine Folge und $|a_k| \leq b_k$ und $\displaystyle \sum_{k=0}^\infty b_k$ 
konvergent, dann ist auch $\displaystyle \sum_{k=0}^\infty a_k$ (absolut) konvergent.

\mysubsection{Quotientenkriterium}

TODO

\mysubsection{Limes von Reihe}

$a = \lim_{n\to\infty} \sqrt{a+2} \rightarrow a = \sqrt{a+2}$

\mysubeqsec{\sum_{k = 0}^\infty \left(-1\right)^k a_n}

\noindent Konvergent, wenn $a_n$ streng monoton fallend (Leibnitz-Kriterium).


\mysection{Grenzwerte}

\mysubsection{Wichtige Grenzwerte}

\mydoubleeq{\displaystyle \lim_{x \to \infty} \frac{ax+b}{cx+d} = \frac{a}{c}}{\displaystyle \lim_{x \to \infty} \sqrt[x]{a+bx} = 1}

\mydoubleeq{\displaystyle \lim_{x \to -\infty} \left(1+\frac{1}{x}\right)^x = e}{\displaystyle \lim_{x \to 0} \frac{\ln(x+1)}{x} = 1}

\myeq{\boxed{\displaystyle \lim_{n \to \infty} \left(1 + \frac{b}{cn}\right)^n = e^\frac{b}{c}}}


\mysubsection{L'Hospital}

Wenn $h(x) = 0$, dann ist $\lim \frac{g(x)}{h(x)} = \lim \frac{g'(x)}{h'(x)}$.

\mysection{Tangenten}

Allgemein Tangentengleichung: $\displaystyle y = mx + b$

\begin{equation*}
	m = \lim_{h \to 0} \frac{f(a+h) - f(a)}{(a+h)-a}
\end{equation*}

Allgemeine Form der Tangentengleichung an der Stelle $a$:

\begin{equation*}
	t(x) = f'(a)\cdot(x-a)+f(a)
\end{equation*}

Die Tangente ist äquivalent zur Taylorreihe ersten Grades der jeweiligen Funktion. $b$ bestimmt sich durch
Gleichsetzung mit $f(x)$ an der Stelle $x$, wo die Tangente erstellt werden soll.

\mysection{Schranken finden}

Die kleinste obere Schranke heißt \textit{Surpremum}. Bei monoton fallenden Folgen ist das erste
Folgenglied die obere Schranke. (Bei monoton steigenden ist das erste das kleinste und daher das 
\textit{Infimum}). Wenn Monotonie weder (s)mf noch (s)ms, dann obere Schranke abschätzen und
per Induktion beweisen (dann nicht notwendigerweise kleinste obere Schranke!).

\mysection{Taylorreihe}

\noindent Allgemeine Formel zur Bestimmung von Taylorreihenapproximationen an der Stelle $x_0$: 
\begin{equation*}
	f(x) \approx \sum_{k = 0}^n \frac{f^{(k)}(x_0)}{k!}\cdot \left(x-x_0\right)^k
\end{equation*}

\mysubsection{Approximationsfehler}

\noindent Der Fehler einer Taylorapproximation an der Stelle $a$ kann abgeschätzt werden mit:

\begin{equation*}
	R_n(x) = \frac{M}{(n+1)!)}(x-a)^{n+1},
\end{equation*}

\noindent wobei $M$ eine obere Schranke von $|f^{(n+1)}(z)| \geq M$ sein muss.

\mysection{Ableitungen}

\mysubsection{Definition Ableitung}

\begin{equation*}
	f'(x) = \lim_{\epsilon \to 0} \frac{f(x + \epsilon) - f(x)}{\epsilon}
\end{equation*}

\mysubsection{Produktregel}

\begin{equation*}
	f(x) = g\cdot h \myto f'(x) = g'\cdot h + g\cdot h'
\end{equation*}

\mysubsection{Quotientenregel}

\begin{equation*}
	f = \frac{g}{h} \myto f' = \frac{h\cdot g'-h'\cdot g}{[h]^2}
\end{equation*}

\mysubsection{Kettenregel}

\begin{equation*}
	f = g(h) \myto f' = g'(h)\cdot h'
\end{equation*}

\mysubsection{Spezielle Ableitungen}

\myeq{\frac{\partial}{\partial x}\left(\frac{a}{(b+cx)^k}\right) = -ack(b+cx)^{-k - 1}}

\myeq{f(x) = ab^{x+c} \myto f'(x) = a\ln(b)\cdot b^{c+x}}

\myeq{f(x) = \ln(x) \myto f'(x) = \frac{1}{x}}

\myeq{f(x) = \sqrt{x} \myto f'(x) = \frac{1}{2\sqrt{x}}}

\myeq{\sin x \myto \cos x}

\myeq{\cos x \myto -\sin x}

\mysection{Integration}

\myeq{\int x^n \dx = \frac{1}{n+1}x^{n+1} + C}

\myeq{\int c\cdot f(x)\dx = c \cdot \int f(x) \dx}

\myeq{\int\left(f(x) +g(x)\right) \dx = \int f(x)\dx + \int g(x)\dx}

\myeq{\int f'(x)g(x)\dx = f(x)\cdot g(x) - \int f(x)\cdot g'(x)\dx}

\myeq{\int \frac{f'(x)}{f(x)} \dx = \ln(f(x)) + C}

\myeq{\int \frac{1}{u^2} \du = \frac{1}{u}}

\mysubsection{Partielle Integration}

\myeq{\int f'(x)\cdot g(x) \dx = f(x)\cdot g(x) - \int f(x)\cdot g'(x)\dx}

Oder auch: $\displaystyle \int u \dv = uv - \int v\du$. Wähle $u$ so, dass es
nach endlich vielen Ableitungen eine Konstante wird.


\mysubsection{Partialbruchzerlegung}

\noindent Echt gebrochene Funktion: $\frac{a^2}{c^3}$\\
\noindent Unecht gebrochene Funktion: $\frac{a^3}{c^2}$\\

\noindent Falls unecht gebrochen:\\
\noindent 1: Polynomdivision (falls unecht gebrochen)\\
\noindent 2: ullstellen des Nenners berechnen\\
\noindent 3: Jeder Nullstelle ihren Partialbruch zuordnen\\
\noindent 4: Ansatz zur Partialbruchzerlegung aufstellen\\
\noindent 5: Koeffizienten bestimmen

\mysubsubeqsec{\frac{f(g)}{g(x)\cdot h(x)\cdot i(x) \cdot \cdots} = \frac{a}{g(x)} \cdot \frac{b}{h(x)} \cdot \frac{c}{i(x)} \cdot \cdots}

$a, b, c, \cdots$ Herausfinden, indem man schaut, ob man durch geschicktes Einsetzen und Umstellen
eine Gleichung der Form $c = a + n$ (mit $c, n$ fest, aber beliebig) herausbekommt. Beispiel:
$g(x) = x - 5$, dann würde man $x = 5$ setzen und den \frq $g(x)$-Teil\flq\ wegkriegen. Damit
$a, b, c, \cdots$ berechnen und dann einzelne Integrale bilden.

TODO!!! Richtiges Beispiel rechnen!!!

\mysection{Differentialgleichungen}

\mysubsection{Matrix-Methode}

\noindent Schritt 1: Differentialgleichungen als Koeffizientenmatrix aufschreiben\\
\noindent Schritt 2: Eigenwerte herausfinden, Eigenvektoren $\displaystyle\vec{v_n}$ bilden\\
\noindent Schritt 3: Lösung ist System aus Gleichungen {\Large $\displaystyle\vec{x} = \sum c_n \vec{v_n} e^{\lambda_n x}$} 


\mysubsection{Determinanten}

$$ \boxed{\det \begin{pmatrix} a & b & c \\
	d & e & f \\
	g & h & i
\end{pmatrix} = aei - afh - bdi + bfg + cdh - ceg} $$

\bmyeq{(a+b)^3 = a^3 + 3a^2b + 3ab^2 + b^3}

\bmyeq{(a-b)^3 = a^3 - 3a^2b + 3ab^2 - b^3}

\bmyeq{(a+b)(a-b)^2 = a^3 -a^2b -ab^2 + b^3}

\bmyeq{(a-b)(a+b)^2 = a^3 + a^2b - ab^2 - b^3}

\bmyeq{x_{1/2} = -\frac{p}{2} \pm \sqrt{\left(\frac{p}{2}\right)^2-q}}

\mysection{Wichtige Summen}

\mydoubleeq{\sum_{n = 0}^\infty \frac{x^n}{n!} = e^x}{\sum_{k = 0}^\infty \frac{g}{a^k} = g\cdot \frac{a}{a-1} \text{ (für } |a| > 1\text{)}}

\mydoubleeq{\sum_{n=0}^\infty (-1)^n \frac{x^{2n+1}}{(2n+1)!} = \sin(x)}{\sum_{n=0}^\infty (-1)^n \frac{x^{2n}}{(2n)!} = \cos(x)}

\mydoubleeq{\sum_{n=1}^\infty \frac{1}{n^2} = \frac{\pi^2}{6}\text{ (Basler Problem)}}{\sum_{n=}^\infty\frac{1}{n^2} = \frac{\pi^2}{6}}

\myeq{\sum_{n=0}^\infty \left(-1\right)^n\cdot z^k = \frac{1}{1+z} \text{ (für }|z| < 1\text{)}}

\myeq{\sum_{k=0}^\infty \frac{a}{b^k} = \frac{ab}{b-1} \text{ (für }|b|>1\text{)}}

\myeq{\sum_{k=0}^\infty \frac{a^{k+1}}{b^k} = \frac{ab}{b-a} \text{ (für }|a|<|b|\text{)}}

\myeq{\sum_{n=0}^\infty \frac{a^{n-1}}{b^n} = \frac{b}{a(b-a)} \text{ (für }|a|<|b|\text{)}}

\myeq{\sum_{n=0}^\infty \frac{a((-b)^n)}{c^n} = \frac{ca}{b+c} \text{ (für }|b|<|c|\text{)}}

\myeq{\sum_{n=1}^\infty\frac{(-1)^{n+1}}{n} = \ln(2) \text{ (Alt. harm. R.)}}

\myeq{\sum_{n = 0}^\infty \frac{(-1)^n}{2n+1} = \frac{\pi}{4} \text{ (Leibnitzreihe)}}

\myeq{\sum_{n=0}^\infty x^n = \frac{1}{1-x} \text{ (Geometrische Reihe)}}

\myeq{\sum_{n=0}^\infty\frac{1}{n} = \infty \text{ (Harmonische Reihe)}}

\myeq{\sum_{n = 0}^\infty (-1)^n \frac{x^{n+1}}{n+1} = \ln(1 + x) \text{(für }x \in (-1, 1]\text{)}}


\[
\displaystyle
\sum_{k = 0}^\infty \frac{a^k}{b^k} = \sum_{k = 0}^\infty \left(\frac{a}{b}\right)^k = \begin{cases} %
	\frac{b}{b-a} & \text{Für }|a| < |b| \\ %
	-\frac{b}{a-b} & \text{Für }\left|\frac{a}{b}\right| %
\end{cases}
\]


\mysection{Extrempunkte}

Bei einem EP ist $f'(x) = 0$. 

Bei $f''(x) > 0$ ist es ein Tiefpunkt, bei $f''(x) < 0$ ein Hochpunkt.

\mysection{Beispielrechnungen}

\mysubeqsec{\int \cos(ax)\dx}

$$ u = ax, \du = a\dx, \frac{\du}{a} = \dx $$

$$ \cos(ax)\dx = \int\cos(u)\frac{\du}{a} = \frac{1}{a} \int \cos(u)\du $$
$$  = \frac{1}{a} \sin(u) = \underline{\underline{\frac{1}{a} \sin(ax) + c}} $$


\end{multicols}

\end{document}
